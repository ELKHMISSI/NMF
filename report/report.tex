\documentclass[11pt,letterpaper,twocolumn]{article}


\usepackage[utf8]{inputenc}
\usepackage[french]{babel}
\usepackage{float}
\usepackage{xcolor}
\usepackage{verbatim}
\usepackage{charter}
\usepackage{amsmath}
\usepackage{appendix}
\usepackage{ragged2e}
\usepackage{array}
\usepackage{etoolbox}
\usepackage{fancyhdr}
\usepackage{booktabs}
\usepackage{arydshln}
\usepackage{caption}
\usepackage{subcaption}
\usepackage{enumitem}
\usepackage{geometry}
\geometry{
  top=0.8in,            
  inner=0.5in,
  outer=0.5in,
  bottom=0.9in,
  headheight=4ex,       
  headsep=6.5ex,         
}
\usepackage{graphicx}
\usepackage{mathtools}
\usepackage{multirow}
\usepackage{pdfpages}
\usepackage{subfiles}
\usepackage[compact]{titlesec}
\usepackage{stfloats}
\usepackage{hyperref}

\setlength{\columnsep}{30pt}


\pagestyle{fancy}
\fancyhf{}
      
\fancyfoot{}
\fancyfoot[C]{\thepage} % page
\renewcommand{\headrulewidth}{0mm} % headrule width
\renewcommand{\footrulewidth}{0mm} % footrule width

\makeatletter
\patchcmd{\headrule}{\hrule}{\color{black}\hrule}{}{} % headrule
\patchcmd{\footrule}{\hrule}{\color{black}\hrule}{}{} % footrule
\makeatother

\definecolor{blueM}{cmyk}{1.0,0.49,0.0,0.47}



\chead[C]

    
\begin{document}
\twocolumn[\begin{@twocolumnfalse}




\centerline{\rule{0.95\textwidth}{0.4pt}}

\begin{center}
    
    %\begin{minipage}{}
    
     \textbf{\huge Application de la factorisation des matrices positives dans un système de recommandation} 

    %\end{minipage}
    
\end{center}

\centerline{\rule{0.95\textwidth}{0.4pt}}

\vspace{15pt}

\begin{tabular}{lr}
    Mohamed EL KHMISSI &\\
    Mohamed Abdelmalek BOUARROUDJ &\\
	Rapport de projet d'Apprentissage Statistique (HAX907X) &\\
	M2 SSD & \\
    2022/2023 & 
\end{tabular}    


\vspace{15pt}
\end{@twocolumnfalse}]
%%%%%%%%%%%%%%%%%%%%%%%%%%%%%%%%%%%%%%%%%%%%%%%%%%%%%%%%%%%%
\section{Introduction}
\par \vspace{1mm}
\justify
 
Ce rapport se situe dans le cadre du projet de module $HAX907X$, dans lequel on va présenter la méthode NMF (Non-negative matrix factorisation) pour construire un système de recommandation, en travaillant sur des jeux de données de \href{https://www.bookcrossing.com/}{$Book-crossing$}, dont le but de recomender des livres aux utilisateurs.

\par \vspace{1mm}  %utilizar estos comando para tener una separación de parrafos adecuada.

%%%%%%%%%%%%%%%%%%%%

%\begin{equation}
%    \label{eq:eg2}
%\end{equation}
%%%%%%%%%%%%%%%%%%%%%
%%%%%%%%%%%%%%%%%%%%%%%%%%%%%%%%%%%%%%%%%%%%%%%%%%%%%%%%%%%%

\section{La méthode NMF}
\justify
La méthode NMF c'est une méthode du filtrage collaboratif, qui consiste à chercher des variables explicatives 
%%%%%%%%%%%%%%%%%%%%%%%%%%%%%%%%%%%%%%%%%%%%%%%%%%%%%%%%%%%%

\begin{thebibliography}{9}
\bibitem{Bordat} P. Bordat et al. ``An improved dimethyl sulfoxide force field for molecular dynamics simulations'' \textit{J. Chemical Physics} Vol. 374 (2003) p. 201-205.
\end{thebibliography} 

\end{document}
